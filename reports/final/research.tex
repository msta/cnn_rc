\chapter{Research}

As shown in \autoref{sec:related_works}, the research community have come up with a multitude of solutions for the RC problems based on neural networks in the recent years. Models are frequently built upon each other. Techniques such as regularization are done with multiple techniques at an experimental level and the effects may overlap to an unknown extent. Finally, restrictions on computational power and time to execute experiments may limit the degree to which results are certain. Modern neural networks used for NLP have lots of randomness in them which influences the test scores. Since they usually also consist of many different layers, small implementation details can make small changes in the results which makes published results harder to reproduce. 

This chapter presents a series of experiments which highlights three problems in the common approach to Relation Classification. We shall examine a common CNN baseline and highlight implementation problems that is caused by reproduceability issues. Then, we look at some simple reductions that can be made to the baseline without reducing general performance. And finally we shall investigate the problem of using the SemEval dataset and its associated practice. 

Each section will presented an individual problem with suggested solutions. 


\section{Baseline Experiment}

As shown in \autoref{chap:related_works} CNNs have shown promising results for relation classification as n-gram selectors. I have chosen to implement a CNN for the SemEval dataset which is based on \cite{re_cnn}. The architecture is simple in that it only uses the input given in the dataset, pre-trained word-embeddings and no additional linguistic resources, but it also presents a number of problems which can be addressed. 

\subsection{Model Architecture}

I now describe the complete model architecture. The architecture consists of five sequential layers:

\subsubsection{Preprocessing}
Each input sentence is sanitized and tokenized. The index of each entity and the distance from each word to the entity words are calculated. 
Finally, sentences which have entity distances greater than a bound $n = 15$ are removed from the training set to reduce noise.


\subsubsection{Embedding layer}
Each word $w^i$ in the sentence is mapped to a $d$-dimensional vector $w_i^d$. The calculated distances to the respective entities $e1$ and $e2$ are mapped to $p$-dimensional embeddings $w_i^{p1}$ and $w_i^{p1}$. These vectors are concatenated to form the word representation $s_i$, which is part of the full sentence representation $S$:

$$
s_i = [w_i^d;w_i^{p1};w_i^{p2}]^T  \textrm{and} S = [s_1, \ldots, s_n]
$$

These vectors are not initialized randomly but loaded from a set of pre-trained word vectors. A word that does not occur in the pre-trained set are initialized randomly and uniformly.  

\subsubsection{Convolutional Layer}

The third layer applies the convolution operator described in \eqref{eq:conv}. Several window sizes are applied at the same level, which represents examining different n-grams in the sentence. Each output neuron in a single convolution produces 
a score $s_{i,j}$ where $i$ is the filter and $j$ is the n-gram. A full convolution with window size $w$ produces $S*^w \in \mathbb{R}^{(n-w+1) \times k}$, where $k$ is the number of filters. The entire layer with an example configuration ${2,3,4,5}$ produces a list of matrices $S*^2, S*^3, S*^4, S*^5$. I shall denote the number of windows as $n_d$

\subsubsection{Pooling layer}

Each convolved window $S*^w$ is max-pooled to produce a score sequence $P^w$. Each filter $i$ in $S*^w$ produces a single score $p_i^w$:
\begin{equation}
p_i^w = max\{S*_{i,1}^w, S*_{i,2}^w, \ldots, S*_{i,n-w+1}^w\} \textrm{and} $P^w$ = [p_1^w, p_2^w, \ldots, p_k^w] 
\end{equation}  
Each window size are concatenated to form a feature vector $z \in \mathbb{R}^k*n_d$. The example configuration from above is then $z = [P^2;P^3;P^4;P^5]$. 

\subsubsection{Fully-connected layer with dropout}

$z$ is input to a regular fully connected layer which measures the contribution of each pooling score to the target relations. Regularization is performed by using dropout on $z$ which outputs $z_p$ where each element is either kept or set to 0 according to a Bernoulli distribution over the hyperparameter $p$.
The activation of this fully connected layer is the softmax function defined in \eqref{eq:softmax}, which outputs a discrete probability distribution for the sentence over the number of target relations.  

\subsubsection{Training}

The network is trained with stochastic gradient descent using shuffled mini-batches. The gradients are computed using the backpropagation algorithm while the weights are updated with the AdaDelta update rule.  


\subsection{Reproduceability}
 
When a chosen is baseline modeled over another described network - and it usually is to allow comparison -  the results from the network given a specific dataset should be very close to the same. However, two common problems are observed in the networks described in \autoref{ch:related_works}.

\subsubsection{Hyperparameters}
 The networks described in \autoref{ch:related_works} suffer from a description of the model that is sufficiently detailed as to allow re-implementation. I highlight a few examples from \cite{re_cnn}:

\begin{itemize}
    \item It is well documented that initial values of the parameters in the network can have significant effect on the results 


\end{itemize}


 The SemEval dataset is measured with the official scorer on a predesignated test split.    

\subsubsection{Measuring the Result}

Gist: Scores are published en-masse with comparisons to other systems. It is hard to reproduce the results and hyperparameters are not properly specified

\subsection{Solutions to increase reproduceability}

\subsubsection{Complete hyperparameter setup}

Assumptions about the input - what is the input, how are the tags processed, what is the tokenization options, are the input clipped to reduce noise.
Embeddings - dimensions, training architecture, using others use a link, dataset specifications
Layers: Bias, initializations, paddings, weight constraints, 
Training: General algorithm, learning rates, regularizations, dropouts, early stopping splits, chosen epochs, loss functions

\subsubsection{Publish minor statistics about test-scores}:
100 runs - or appropriate runs for the dataset. Min, max, mean and stddev. 
Why is this more useful? Better comparisons. Also allows a potential implementer of the system
to choose if you want a system with less variance but a lower mean. 
T-test as alternative for the stats. 


\section{Unmotivated complexity}

\subsection{Case study: CNN}

\subsubsection{Positional Embedding or Embedding Indicator?}
Show that the positional embedding can be reduced to 1 without any loss in mean score. 

\subsubsection{L2 Normalization or Nothing at all}
Show that the l2 normalization does not matter and can be removed.

\subsection{Solutions for complexity}

\subsubsection{Reduce capacity of model}
Every time a new feature is added to the network, the hyperparameters should be re-tuned. Sensitivities in the input (markup) influence effectiveness of other parameters. Embedding dimensions hugely influence the number of parameters in the system and should be kept at a bare minimum. 
Have a clear motivational path for each part of the network to its theoretical grounding. I'm not a fan of ``well its a trick that works well'' strategies like the L2 normalization.

\subsubsection{Use semi-supervised approaches}

We could use more data. Learning curve here:


Use it as regularizer: Only get sentences above a threshold that has same entity pairs as the training set. This almost guarantees candidate selection

Use it to find a new signal: harder. Tried different datasets with no good results. 
Reason: The dataset have a hard time generalizing to new data.  

TODO

make 3 folders
stick text in for each problem
and figures


\subsection{Overfitting as a community}

\subsubsection{Problems with SemEval}

.. How do we actually prove this? Perplexity from SemEval -> any other data set is poor. Also the train-test perplexity is relatively low.

\subsubsection{Generalizes poorly}

The entity selection process is obscure and hard to apply in general. Getting new entities is not easy ... ?
Idea: Do a run of the experiment without pos embedding, and then observe self-training.?

\subsection{Solutions to the SemEval problem}

We are probably bound somewhat to the dataset due to the rarity of quality relation classification datasets.

\subsubsection{Dont use the official test-train split}
Generate a random test-train split from the data. The same samples are not interesting, not even for comparison.

\subsubsection{Dont use the official scorer}
Use test loss because it assumes less about the actual meaning of the systems. 
Designing a system for semantic relation classification, not for semeval relation classification :)

\subsubsection{Dont use SemEval without other tasks}

The model should be applied to another similar tasks. The dataset sune showed is a good example.
Effects on early stopping and the setup in general. 









