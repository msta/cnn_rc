
\chapter{Preliminaries}

This chapter provides the background information necessary understand the relation classification task and how the task can be solved with machine learning. 

\section{Task Definitions}

Natural Language Processing (NLP) is a research area which combines linguistics and computer science with the 
goal of understanding human - or natural - language in various forms such as written text or audio recordings. We might separate the field further in smaller challenges like Natural Language Understanding (NLU) or Natural Language Generation (NLG), but for the rest of this report I will present all related tasks under the definition NLP.\\

Understanding human language in a computer usually involves building a structured model over the input language which the computer can understand and interpret. These structures can model low-level syntactic information about the structure of the language. An example of such as model is a \emph{Part-Of-Speech (POS) tagger}, which labels each word token with a distinct category related to the definition of the word and its context. The output of a POS-tagger can be used as input in another task. An example of such a task is the \emph{semantic classification task}, which captures higher-level semantic information. I will cover both definitions.

\subsection{POS-tagging}
\label{pos-tagging}
TBD
Given a sequence of written text as input we can label each word 

\begin{center}
`` She soon had a stable of her own rescued hounds ''
\end{center}


\subsection{Semantic Relation Classification}
\label{define_rc}
With the output of the task defined in \ref{pos-tagging} we can now look for pairs of certain entities in sentences and extract their relation. Specifically, we can look for pairs of common nominals and identify a potential semantic relation between them. Common nominals are word(s) which act as a noun in a sentence. Identifying and categorizing these relations is called \emph{semantic relation classification}. The difference between general relation classification and semantic relation classification is subtle and not well defined. The word `semantic' is used because of the possible classes the relation classification task can output\cite{semeval2007}\cite{semeval2010}. For the rest of the thesis, I will denote semantic relation classification as simply relation classification (RC). \\
The goal of the task is to output an ordered tuple which conveys meaning about the relationship between the marked nominals.
Consider the example used above where the common nouns have been marked:

\begin{center}
`` She soon had a stable$_{e1}$of her own rescued hounds$_{e2}$''
\end{center}

A correct output in this example is the tuple Member-Collection(e2,e1). The label is intended to capture subtle semantic information about the nominals. A "stable" can mean a physical building which houses animals, but it can also mean a set of animals which are grouped together by human means. The Member-Collection tuple defines the word "stable" to have a specific meaning due to the relation to the other nominal.\\

The task can be more formally defined:
Given an input sentence $S$ and entities $W_{e1}$ and $W_{e2}$ which are taken from $S$, we define an input space $\mathcal{X}$ which is the set of all triples $(S,W_{e1},W_{e2})$. %(THAT CAN BE GENERATED?)  
We also define a set of labels $\mathcal{Y}$. These labels are probably different from dataset from dataset, but they define the space of the relations we are looking for. The relation classification task is to find a mapping $\mathcal{H} : \mathcal{X} \mapsto \mathcal{Y}$. The correct mapping $\mathcal{H}^*$ is not well-defined. Generally the requirement for $\mathcal{H}$ is that some human evaluator must agree with the semantic relation which is given by $\mathcal{H}$. When solving the task for a specific dataset, a test set can be provided with a given (sentence,label) set. The goal of RC is then to output the same labels as the human annotators.

A closely related task to RC is \emph{relation extraction} (RE) . The actual difference between the two tasks are ill-defined and often confused because 1) they are both classification problems and 2) they both have the same inputs. For the sake of clarity in this thesis, I will define the RE task to include first the binary classification of whether an actual relation exists between inputs. This means that the RE tasks usually use unbalanced datasets, where the majority of the samples have no relation such as the ACE 2005 dataset\cite{ace2005}. 

Finally, RC assumes that the common nouns have always been correctly identified. This assumption, of course, does not hold if POS-tagging is done by a statistical algorithm. Incorrect labels in the output might lead to error propagation.
A way to deal with this assumption is to include the marking of the common nominals in the RC task. This approach is called \emph{end-to-end relation classification}, or \emph{relation detection}. This approach lies beyond the scope of this thesis\\



\section{Datasets}

One major problem with the RC task is the lack of quality datasets that can be used in academia to measure the performance and robustness of freshly developed classifiers. Relations are both hard to define and classify, and the diversity of language usually requires that a large amount of samples are labeled before the dataset is useful. The types of relations which the model should classify also varies greatly. Relations are not easy to define generically and are sometimes even defined only for a specific research paper\cite{culotta_re}\cite{med_re}. Finally, to use the dataset on the relation classification task, the dataset should be

\begin{itemize}

\item \emph{Open}: The dataset should be publicly available so the research community can use it without too many restrictions, and claimed results on the dataset can be validated

\item \emph{Well-defined}: The format of the dataset should be well-defined to avoid large amounts of preprocessing which can have significant effect on the subsequent classification. It can also be useful if some official metric is defined for measuring the performance on the dataset. While the goal of a classification model may vary on the dataset, having an official tool helps the research community compare the results of their models. 

\item \emph{Clean}: This requirement is specific to the relation classification task. The relation extraction task should cover gathering the input for the classification. As such, noisy and missing input misses the goal of the RC task. 

\item \emph{Used}: Finally, the dataset is most useful if it has some traction in the research community, which creates a well-defined benchmark for the task. 

\end{itemize}

\subsection{SemEval 2010 Task 8}

One dataset which shows the characteristics listed above is the dataset created for the 2010 SemEval task 10\cite{semeval2010}. This dataset defines a sentence classification task with relation labels which are semantically exclusive. The task is canonical for relation classification as the distribution of labels are fairly equal, which means that the entity detection step is assumed. Otherwise, there would a significant amount of samples in the dataset with no relation. Notice that the ``Other'' relation is distinct from a no-relation class. The Other class consists of samples where the type of the relation could not be determined though they still represent some relation.   

\begin{table}
\parbox{.45\linewidth}{
\centering
\caption{Train-test distribution for SemEval 2010 task 8}
\label{my-label}
\begin{tabular}{ll}
Samples &       \\
Train   & 8000  \\
Test    & 2717  \\
Total   & 10717
\end{tabular}
}
\hfill
\parbox{.45\linewidth}{
\centering
\caption{Label distribution for SemEval 2010 task 8}
\label{my-label}
\begin{tabular}{ll}
Relation           & Distribution \\
Cause-Effect       & 12.4\%       \\
Component-Whole    & 11.7\%       \\
Entity-Destination & 10.6\%       \\
Entity-Origin      & 9.1\%        \\
Product-Producer   & 8.8\%        \\
Member-Collection  & 8.6\%        \\
Message-Topic      & 8.4\%        \\
Content-Container  & 6.8\%        \\
Instrument-Agency  & 6.2\%        \\
Other              & 17.4\%      
\end{tabular}
}
\end{table}



In summary, solving the RC task requires us to find a general function which can label sentences by only knowing the entities and the sentence itself. To find this function we turn to a branch of computer science called \emph{machine learning}.



\section{Machine Learning}

Modern solutions to the RC problem is almost always based on machine learning techniques. These techniques allow computers to solve problems by learning rules and patterns from data. An alternative to machine learning is hand-written rule-based systems - but these are hard to engineer and error-prone since language varies greatly and words have different meaning based on their context. In this section I will describe different learning problems and relate it to the RC problem.  


\subsection{Supervised and unsupervised learning}
A common goal in machine learning is a generalized version of the goal defined in RC: we want to learn a mapping $\mathcal{H} : \mathcal{X} \mapsto \mathcal{Y}$ from a set of datapoints $D_{train}$. The nature of $D_{train}$ and the range of $\mathcal{Y}$ defines the associated machine learning task\cite{semisupervised_book}. $D_{train}$ can be points $(x_i, y_i)$ where $y_i$ is the given label for $x_i$, usually annotated by a human, or it can be unlabeled points $(x_i)$ where no label is given. An algorithm that uses labels $(x_i, y_i)$ to learn $\mathcal{H}$ places itself in the class of \emph{supervised learning algorithms}, while algorithms that only use unlabeled data $(x_i)$ are called \emph{unsupervised learning algorithms}. 

\subsection{Learning problems}

We can define a set of four problems by combining the input $D_{train}$ with the range of $\mathcal{Y}$. If $\mathcal{Y}$ is discrete-valued and the input uses labeled data, we call $\mathcal{H}$ a \emph{classification function}. If $\mathcal{Y}$ is continuous, the task is called \emph{regression}. If we have no labels from the input but Y is still discrete, we call the problem \emph{clustering} as the algorithm usually must make its discrete values. And finally, if no labeled data is available and $\mathcal{Y}$ is continuous, $\mathcal{H}$ can be called a \emph{density estimation function}, since we can interpret $\mathcal{Y}$ as being proportional to a probability distribution over $\mathcal{X}$.

\subsection{Semi-supervised classification}

We can look at the amount of data in $D_{train}$ which have labels to define a broader category of algorithms that contains both supervised and unsupervised algorithms. With this perspective, supervised learning are algorithms that assume all data is labeled, while unsupervised assumes none. If we have \emph{some} data which is labeled,    


\subsection{Self-Training}

\subsubsection{Parameters and Techniques}

\subsection{Metrics}

To evaluate the performance of a RC function $\mathcal{H}$, we can apply the function on a previously unseen set of data $D_{test} = [(S_1, W_{1,e1}, W_{2,e2} \ldots (S_n, W_{n,e1}, W_{n,e2}]$ in which the labels $\mathcal{H}^*(D_{test}) = [y_1 \ldots y_n]$ are known. Since the labels of the relations are discrete, a sample triple $(S_i,W_{i,e1},W_{i,e2})$ is correct if $\mathcal{H}(S_i,W_{i,e1},W_{i,e2}) = y_i$. 
It is also useful to measure what label is predicted instead of the correct label when the classifier is wrong.
By observing the prediction of each label we can construct a \emph{confusion matrix} which shows predictions on one axis and the actual labels on the other. Below is shown an example of a confusion matrix for a 10-way classification problem:

\begin{center}
\fbEpsfig{confusion_matrix}{\textwidth}{htbp}
\end{center}

From the positions in the confusion matrix predictions for each class $Y_i$ are measured and put into four categories: 

\begin{itemize}
\item A \emph{true positive} (tp) are a sample $D_i \in D_{test}$ where $\mathcal{H}^*(D_i) = Y_i$ and $\mathcal{H}(D_i) = Y_i$. 
\item If $\mathcal{H}^*(D_i) \neq Y_i$ and $\mathcal{H}(D_i) = Y_i$, $D_i$ is a \emph{false positive} (fp). 
\item Conversely, if $\mathcal{H}^*(D_i) \neq Y_i$ and $\mathcal{H}(D_i) \neq Y_i$, $D_i$ is a \emph{true negative} (tn).
\item Finally, if $\mathcal{H}^*(D_i) = Y_i$ and $\mathcal{H}(D_i) \neq Y_i$ $D_i$ is a \emph{false negative} (fn).\\ 
\end{itemize}

\subsubsection{F1 Measure}
The accuracy for a multi-class problem is defined with these terms over the number of classes $m$ as $\frac{\sum_{i}^{m} tp(Y_i)}{n} $, but it is not very useful since it does not take the distribution of classes into account. In the relation extraction task, for example, the number of samples which have no relation will greatly outweigh the relevant samples. A classifier may be encouraged to simply label all samples as having no relation, which will yield a high accuracy. Instead, two measures which shows performance for each can be used. \emph{Precision} (p) is defined as $\frac{tp} {tp+fp}$ and indicates how certain a classifier is that predictions for a class actually belong to that class. \emph{Recall} (r) is defined as $\frac{tp}{tp+fn}$ and indicates how sensitive a classifier is to samples belonging to a certain class. Precision and recall are ends of a spectrum - a classifier can achieve maximum precision for a class by never predicting that class, but at the cost of recall. Likewise recall can be trivially obtained by cutting precision. To output a single number for a class which balances these two, the harmonic mean of precision and recall is defined as the \emph{F1 score} $\frac{2 * p * r}{p + r}$. An example of the measures are shown below, drawn from the confusion matrix:

\begin{center}
\fbEpsfig{f1_values}{\textwidth}{htbp}
\end{center}

And finally an averaging strategy for the individual F1 scores must be chosen to output a single score. By averaging each final f1 score we obtain the \emph{macro} F1: 

$$ 
F1_{macro} \frac{\sum_{i}^{m} \frac{2 * p_{Y_i} * r_{Y_i}}{p_{Y_i} + r_{Y_i}}}{m}. 
$$

Alternatively we can sum individual precision and recall values which will value larger classes higher and obtain the micro:

$$
p_{micro} = \frac{\sum_{i}^{m} tp_{Y_i} } {\sum_{i}^{m} tp_{Y_i}+\sum_{i}^{m} fp_{Y_i}} \\
r_{micro} = \frac{\sum_{i}^{m} tp_{Y_i} } {\sum_{i}^{m} tp_{Y_i}+\sum_{i}^{m} fn_{Y_i}} \\
F1_{micro} = \frac{2 * p_{micro} * r_{micro}}{p_{micro} + r_{micro}} \\
$$
For NLP tasks, the macro is often chosen as classes with low frequency are important. 

\subsection{Validation}

Before choosing how to approach the task of finding a solution to the RC task, we have to investigate the effects of choosing a mapping $\mathcal{H}:\mathcal{X}\mapsto\mathcal{Y}$ by learning from $D_{train}$. 


\section{Neural Networks}

There are many different algorithms to choose from when trying to learn a relation classifier. A widely used datastructure is the \emph{neural network}, which have been used extensively in recent years and have yielded significant results. The recent work on neural networks in NLP and specifically RC will be covered in \autoref{related_works}. This section presents the definition of a neural network and how it is trained to approximate $\mathcal{H}$ through derivatives. Then convolutional neural networks are presented since they are relevant for the RC task and other NLP problems.    

\subsection{Definition}

A neural network is a datastructure which can be used to learn a function $\mathcal{H} : \mathcal(X) \mapsto \mathcal{Y}$ from a set of training data $D_{train}$. Usually, the term neural network also encompasses the algorithms which are used to build the datastructure and adjust it to the learned function. A common way to describe neural networks is to describe each word:

The word `network' is used because the datastructure represents a series of functions which are chained together to form a network. This layering of functions forms a directed graph $f(0) \mapsto f(1) \ldots f(n-1) \mapsto {f-n}$ which begins with the input space $\mathcal{X}$ and ends with the output $\mathcal{Y}$.  
Each layer in the network $f(i)$ receives input from the previous layer $f(i-1)$. The first layer $f(0)$ is called the \emph{input layer}, while the final layer is called the \emph{output layer}. Layers in between are called \emph{hidden layers}. The layers are called hidden because they are not defined by the training data, but must be learned by the individual network\citep[chapter 6]{dl_book}. The number of hidden layers are called the \emph{depth} of the network, while the number of neurons in the individual layer is called the \emph{width}. \\

The word `neural' is used to describe these structures because they take inspiration from models of biological brains. As an example, consider a network where the input is a vector $v$. We can apply an entire vector-to-vector function $f(v)$ which outputs a new vector, the input for the next layer. We can also deconstruct $f(v)$ to its individual components:

$$ f(v) =  [f_{v}^{0};f_{v}^{1}\ldots;f_{v}^{n}]^T $$ 

where $f_v^{i}$ act in parallel on the input vector. These components are now vector-to-scalar functions. The output of such a subfunction is inspired by a \emph{neuron} or a \emph{unit} in the brain. The strength of the connection between an input and a neuron is called a \emph{weight}. All weights for a particular neuron are summed to form the \emph{activation} of the neuron. The activation is then a linear combination of the inputs determined by the weights. Finally, the activation is passed through an \emph{activation function} which is also inspired by how neurons ``fire'' in the brain. 

\subsubsection{Activation Function}

The activation function, which is inspired by neuron activation, is applied on the above mentioned activation to introduce non-linearities to the network. The functions usually squeeze the input in a certain interval which is analogous to neurons ``firing'' in the brain when they have received enough signal. As I will show in \autoref{nn_optimization} the derivative of the activation function is an important factor in choosing which activation function to use.

The traditional activation function is the \emph{logistic sigmoid} function, which is a smooth ``S''-shaped function which centers around zero. The derivative is simple and easy to compute. However, the sigmoid suffers not being centered around zero, which is a problem for gradient-based optimization.\citep[p. 66]{dlbook}. To compensate for this problem, the \emph{hyperbolic tangent} function is used, which is centered around zero. Both functions suffer from \emph{saturation} which is also a problem for optimization \cite{activations}. A third option is then the \emph{rectified linear unit} (ReLU), which does not saturate. All three and their derivatives are shown below:

\begin{center}
\fbEpsfig{activation_functions}{\textwidth}{htbp}
\end{center}

% sigmoid $\frac{ds}{dx} = S(x) * (1 - S(x))$

\subsubsection{Bias parameter}




%Notice that the function $\mathcal{H} : \mathcal{X} \mapsto \mathcal{Y}$ will be equal to () 

\subsection{Deep Feedforward Neural Network}

The \emph{deep feedforward network} is the central structure of the neural network topology. Like in the above definition, explaining the structure is usually easiest by breaking down the name.\\

A feedforward network is called \emph{deep} because it has multiple hidden layers which increases the capacity of the model. While it is true that a single layer hidden network can approximate any function, it has been shown that increasing the depth greatly reduces the number of neurons needed in the network compared to a single layer network (INSERT REFERENCE).

A neural network is \emph{feedforward} if it has no connections that leads back to an earlier node. The information flows in one direction through the network when it is being activated. A network that has information from the output flowing back into the network is called a \emph{recurrent neural net}. 


\subsection{Optimization}
\label{nn_optimization}

\subsection{Regularization}

\subsubsection{L2 Constraints}

\subsubsection{Early Stopping}

\subsection{Word Embeddings}
\label{sec:word_embeddings}

Lorem ipsum dolor



\subsection{Convolutional Networks}

This section presents the relevant theory on convolutional networks as they are applied to the task of relation classification. I begin by describing the core theory of convolutions and then move on to how they are applied in NLP. 

Convolutional Neural Networks (CNN's) are a relatively novel neural network structure (LECUN REFERENCE) that can be used to process grid-like input such as fixed-length sentences, images, signals or bounded time-series. The central operation in a CNN is the convolution, and then almost always following the convolution is the pooling operation, both of which will be described below. Notice that there is an important difference between the concepts of convolution and a CNN. The term CNN is used to describe any network that contains a convolution operation, while a convolution is a specific linear transformation. 

\subsubsection{Definitions}

I now describe the central operators of convolutional networks, the \emph{convolution}, the \emph{non-linear transformation} and the \emph{pooling} operation. 


\paragraph{Convolution operator}

The first operator in a convolutional network is called a convolution. The convolution is a general mathematical operation defined on two functions $f$ (also called the input function) and $g$ (also called the kernel function) which produces a third function:
$$
(f * g) (x) = \int_{-\infty}^{\infty} f(X)w(x-X)dX
$$ 

Since this thesis is about machine learning with neural networks, I will restrict the convolution to be defined on discrete input. This means that we don't need to compute the area under the function of the kernel, but rather can sum the individual values:

$$
(f * g) (x) = \sum_{-\infty}^{\infty} f(X)w(x-X)
$$

The output of the function $(f * g)$ is called a feature map, or filter map. Correspondingly, when the output is multi-dimensional, a single dimension will be called a feature or a filter.\\

The above form of the convolution is still not quite relatable to machine learning applications. Usually the input and output of neural network layers are vectors or matrices of real-valued numbers. Text representations can be vectors of words in a sequence, which produces a matrix. To use the convolution in a neural network, I simply define $f$ and $g$ to be functions of the input layer $x$ which produces the input value and the kernel value, respectively. The functions are defined to be zero at any position where the input layer is not defined. 
If the input of $f$ and $g$ produces real numbers, the input of a network layer will be a vector. The convolution is now defined as a finite summation over the elements of the layer. And finally, I will introduce an intercept term $b$ which can shift the linear transformation away from the origin: 
$$
S*(x) = (f*g)(x) = \sum_X f(X) * g(x-X) + b
$$

I use the naming $S*$ for the convolution because of its application to NLP in the thesis - usually the convolution is over a sentence $S$.
This definition of a convolution will be used in my convolutional networks in the rest of the thesis.

\paragraph{Non-linear transformation}

Since the convolution is a linear transformation, it is necessary to apply a non-linear transformation in order to be able to approximate non-linear functions in the network\cite{any_function}. Commonly used functions are the sigmoid, tangent or rectified linear unit (ReLU). The non-linear transformation in CNNs is also called the \emph{detector stage}. The non-linearity can be written conveniently on top of the convolution:

$$
S*(x) = (f*g)(x) = g(\sum_X f(X) * g(x-X) + b)
$$

where $g$ is the activation function. 


\paragraph{Pooling operation}

The last standard operation in a typical layer of a CNN is called the pooling operation. The goal of the pooling operation is to make the output of the convolution \emph{translationally invariant} to small changes in the input. In the context of convolutions as feature detectors, the pooling operation can be thought of as a summarizer of a region of outputs for a certain feature. The operator ``pools'' over a spatial region of output and summarizes their responses. There are many functions that can be used, including:

\begin{itemize}

\item Max-pooling, which simply selects the output from the pooling region with the highest score.
\item Average-pooling, which averages all the outputs from the pooling region.
\item Weighted average, which weighs the average operation with a metric, usually the distance from the central unit in the pooling region.
\item Using the L2 norm of the entire pooling region

\end{itemize} 

When the 
This rate of reduction is called \emph{downsampling} and can improve the efficiency of the network, since the next layer in the network can have fewer inputs while still detecting important features. Below is shown two examples of 
max pooling operations:





The first figure shows a layer of convolved outputs where there is no downsampling, and a pooling region of 2. Here, the max pooling have the effect of ``overriding'' lesser important features with the most important. The second figure shows a global max pooling on the same convolution. Here the intended effect is to select the most important feature of the entire filter. Downsampling and pooling region is set to the entire dimension of the filter. This operation is most useful for single-label classification because it will only select the most important feature. 




\subsection{Motivation}

The motivation for convolutional neural networks comes from both biologically inspired theory as well as practical problems with deep feedforward networks.\\ 

While it may be the case that the actual design and implementation of a human brain is poorly understood, the idea of applying convolutions in neural networks are undoubtedly inspired from biological principles \cite[p.~353]{dlbook}. Since the development of CNNs has been a lengthy process, I will only provide a few examples of features of CNNs that are inspired by neuroscience. \\

\begin{enumerate}

\item CNNs are inspired by areas of the brain such as the primary visual cortex (V1). The V1 is arranged as a topological map, which mirrors the input of the eye as it is captured in the retina. One output in convolution can be interpreted as an output from a locally connected neuron in such a map. The neurons on the left and right shares parameters with this neuron. A layer of these neurons forms a kernel in the convolution. By using a CNN to model the V1, it is possible to achieve state-of-the-art predictions on how the V1 responds to visual input\cite{cnn_sensory_coding}. \\

\item The concept of early layers in the model representing simple features or hidden classes of later units in the model is also directly inspired from the concept of simple cells in the human brain \cite{cnn_taxonomy}. Generally, a CNN models simple cells as detector units (the affine transformation). As an example, simple cells are analogous to edge detectors in image object recognition tasks.\\

\item Similarly, complex cells represent feature detectors that are invariant to small changes in the input. If CNNs are used for recognizing relations or objects in the input, it is advantageous to use a structure that is designed to recognize if some feature is present rather than specifically where. Generally, a CNN models complex cells as the pooling operation which is applied after the detectors. As an example, specific neurons are believed to associate to high-level concepts such as famous individuals \cite{visual_rep_brain}\\
 
\end{enumerate}

These biologically inspired features provide some clear practical advantages over regular feedforward networks in many practical applications. 
Traditional feedforward networks are usually fully-connected on each layer. This is problematic when the input grows as the amount of parameters which must be learnt by the network grows linearly. Secondly, when the task of the network is to detect specific features in the input, sharing parameters across the inputs is a way to reduce the number of parameters while still preserving the ability to detect such features. This means that the same entire kernel will be used on all the inputs. 
As an example, a RGB image object recognition network with 1 hidden layer, prediction of 50 different objects and input image sizes $200 \times 200$ will have $200 * 200 * 50 * 3 = 6,000,000$ weights.    
With CNNs we can design the kernel to be much smaller than the input which greatly reduces parameters while still detecting meaningful features in a practical application. A typical first kernel of a CNN can be as small as $5x5x3$, where 3 is the number of channels in the input image\footnote{\url{http://cs231n.github.io/convolutional-networks/}}. With parameter sharing, the network will have $5*5*3*50 = 3750$ weights, which is an immense reduction. 


\subsection{Convolutional Neural Networks in Natural Language Processing}

I now turn to discuss how CNNs can be applied to text processing and specifically sentence-level problems, which include relation classification. It is useful to consider that a common feature of NLP systems are $n$-grams, which is a tool to capture semantic information between combinations of words that are next to each other. The convolution described below can be seen as a neural network based analogy for learning important n-gram features that are then used for classification at a later stage in the network. 

In \autoref{sec:word_embeddings} I described how words can be represented as dense vectors which represent semantic information about the words. This semantic information can be transferred from other learned tasks or they may be specific to relation classification (such as positional distance to the entities in the sentence).
Assume that we use a vector representation $s$ of size $m \in \mathbb{Z}$ for each word.
A sentence $S_n$ consisting of $n$ words $s_1, s_2 \ldots s_n$ is then expressed as matrix $M \in \mathbb{R}^{n \times m}$. This matrix can be the target of a convolution. The following considerations and design choices must be taken into account when designing the convolutional kernel for NLP:

\begin{itemize}
\item The kernel should have height $m$ so the convolution is over entire words.

\item The width of the kernel represent a specific n-gram and is also called the \emph{window size}.

\item The depth (also called number of filters) of the kernel determines the number of hidden classes or features that can be detected. 

\item Differently from convolutions of images, the stride is usually only one. If the stride increases, some n-gram which is important for the classification step may be omitted. In CNNs used for image processing this is alleviated by the pooling strategy.

\item Since the kernel convolves over entire word vectors an appropriate pooling strategy must be chosen. Max-pooling can be used and represents selecting the most important n-gram for each filter used in the convolution. 

\item Finally a padding must be selected for convolution. Choosing a valid padding reduces the number of n-grams which are built only on padding tokens and is therefore considered noisy. A drawback of valid padding is that it reduces the number of convolutions which can be stacked in a deep network. However, convolutions in NLP are usually done in a single layer and so this is not important. A valid padding is usually chosen.

\end{itemize}   

Recent work on relation classification with CNN's follows the above guidelines \cite{att_cn}\cite{re_cnn}\cite{cnn_rank}. Usually, the network is not deep in the number of convolutions, but it can have several convolutions at the same level to detect different sizes of n-grams. Following the above guidelines and using $k$ filters for each window size $w$, the output matrix of such a convolution is $S* \in \mathbb{R}^{n-w+1 \times k}$. Since the convolution is almost always followed by a non-linear transformation, I write that transformation directly in $S*$ so that:

$$
S*_{i,j} = g( \sum_{x=0}^{w-1} f_{x+1,j}^{T} f_{x+i,j}^{T} + b_j)
$$

Where $b$ is an applied bias $\in \mathbb{R}$ and $g$ is a non-linear function such as the ReLU, tangent or the sigmoid function. 
To clarify further, $j$ indexes into the specific filter and $i$ indexes a specific n-gram. Consequently each filter has a vector $s_j = [S*_{1,j}, S*_{2,j} \ldots S*_{n-w+1,j}]$ which represents detecting a specific hidden class from all the n-grams in the sentence. The entire kernel is learnt by the CNN with training. 
Finally the pooling step is applied to $S*$ to achieve translational invariance and select the most important n-grams for each hidden class. The pooling operation is a global max pooling and produces a vector:

$$
P_{S*} = [max\{S*_{1}\} ; max\{S*_{2}\} \ldots max\{S*_k\}]
$$

where $S*_{i}$ is the $i$'th filter of the convolution.












\input{data_augmentation.tex}


